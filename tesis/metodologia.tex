


\section{Metodologia}

%\begin{table}[h!]
\centering
\setlength{\arrayrulewidth}{0.8pt}
\setlength{\tabcolsep}{5pt}
\renewcommand{\arraystretch}{1.2}


\small
\centering\begin{tabular}{|p{9cm}|*{11}{c|}}
\hline
\textbf{Atividades} & \multicolumn{11}{c|}{\textbf{Meses}} \\
\hline
  &
\textbf{1} & \textbf{2} & \textbf{3} & \textbf{4} & \textbf{5} & \textbf{6} & \textbf{7} & \textbf{8} & \textbf{9} & \textbf{10} & \textbf{11} \\ 
\hline
Criação de uma pipeline de dados para o armazenamento e tratamento dos dados & x & x &   &   &   &   &   &   &   &   &   \\ \hline
Análise detalhada das técnicas em sistemas CAD                              &   & x &   &   &   &   &   &   &   &   &   \\ \hline
Construção de uma prova de conceito do modelo de CNN para avaliar a viabilidade das técnicas escolhidas &   & x &   &   &   &   &   &   &   &   &   \\ \hline
Análise dos resultados da prova de conceito do modelo e definição das técnicas para a construção do modelo &   &   & x &   &   &   &   &   &   &   &   \\ \hline
Construção do modelo de CNN                                                &   &   &   & x & x &   &   &   &   &   &   \\ \hline
Avaliação do modelo de CNN e ajuste de hiperparâmetros                     &   &   &   &   & x & x &   &   &   &   &   \\ \hline
Construção da prova de conceito de um modelo de MLP                        &   &   &   &   &   & x &   &   &   &   &   \\ \hline
Análise de resultados da prova de conceito do modelo de MLP e definição das técnicas para o desenvolvimento do modelo de MLP &   &   &   &   &   &   & x &   &   &   &   \\ \hline
Construção do modelo MLP                                                   &   &   &   &   &   &   & x & x &   &   &   \\ \hline
Testes com o modelo e ajustes de hiperparâmetros                           &   &   &   &   &   &   &   & x & x &   &   \\ \hline
Análise de precisão do modelo em relação ao modelo de CNN                  &   &   &   &   &   &   &   &   & x & x &   \\ \hline
Análise de viabilidade do sistema                                          &   &   &   &   &   &   &   &   &   & x &   \\ \hline
Elaboração de artigo científico com apresentação dos resultados            &   &   &   &   &   &   &   &   &   & x & x \\ \hline
Processo de submissão do artigo científico para periódico                  &   &   &   &   &   &   &   &   &   &   & x \\ \hline
\end{tabular}
\caption{Cronograma de atividades do desenvolvimento do projeto.}
\label{cronograma}
\end{table}





Presente projeto tem característica quantitativa, visto que se enquadra nas definições de ser conseguido na busca de resultados exatos evidenciados por meio de variáveis preestabelecidas, em que se verifica e explica a influência sobre as variáveis, mediante análise da frequência de incidências e correlações estatísticas \cite{michelmetodologia}.

O desenvolvimento do trabalho se divide em duas grandes áreas de atuação, Inteligência artificial, que atende todo o escopo do desenvolvimento do modelo de segmentação. E engenharia de software, que engloba os conteúdos necessários para o desenvolvimento do protótipo de software.

\subsection{Do desenvolvimento do modelo}

Conforme apresentado na seção anterior, a aplicação das técnicas de visão computacional na medicina tem se mostrado muito promisoras para o diagnóstico de doenças complexas e melhora na eficiência do diagnóstico clínico. A integração de diferentes dados, como exames laboratoriais e imagens médicas, possibilitará a construção de um sistema robusto, capaz de analisar diferentes tipos e grandes volumes de dados. A Figura \ref{fig:pipeline} ilustra, de forma abstrata, como será o funcionamento do sistema proposto.

\begin{figure}[h]
    \centering
    \includegraphics[scale=0.15]{images/pipeline2.png}
    \caption{Pipeline do projeto}
    \label{fig:pipeline}
\end{figure}

Cada fase do pipeline será projetada para realizar operações críticas para o resultado do modelo, buscando também maior desempenho e resiliência para esse sistema. Serão detalhadas cada uma das etapas.

\subsubsection{Coleta dos dados}

Essa etapa é responsável pela extração e organização dos dados, serão utilizadas técnicas de engenharia de dados para garantir que todo o processo de coleta dos dados seja seguro e consistente. Desta maneira, para assegurar esses objetivos, o sistema terá integração com dados de diversas fontes diferentes, essa integração para a coleta das informações será feita através de interfaces de programação de aplicação (\textit{API}), que possibilitará a automatização dessa fase. Para o desenvolvimento dessa etapa, serão empregadas a linguagem de programação python e a ferramenta \textit{spark} como recursos principais. No contexto deste estudo, a coleta foi realizada a partir de bancos de dados públicos médicos, o portal escolhido responsável pelas informações é o The Cancer Imaging Archive \cite{cancerimaging}, foram extraídos diferentes conjuntos de dados. 

A figura \ref{fig:ex-dataset}, apresenta uma imagem retirada do conjunto fornecido pelo the Cancer Imaging Archive. Esse conjunto é composto por 610 imagens microscópicas de melioma múltiplo. A localização do câncer é nos ossos, as imagens possuem ampliação de 1000x, tendo as dimensões de 1126x874 pixels.


\begin{figure}[h] % Ajuste largura
    \centering
    \includegraphics[width=0.3\textwidth]{images/datasetexample.jpg}
    \caption{Imagem microscópica de câncer nos ossos}
    \label{fig:ex-dataset}
\end{figure}

\subsubsection{Pré processamento}

Após a etapa anterior, será necessário tratar as informações coletadas. Com isso, será aplicado operações como a limpeza de dados, que envolve a remoção de valores inconsistentes, ajuste de escalas de valores e codificação de variáveis categóricas, por exemplo. Além disso, para os dados de imagens, serão utilizadas técnicas de aumento de dados, para maior generalização do modelo. Também serão explorados técnicas de processamento digital de imagens, como regulagem de luminosidade e transformação para diferentes domínios de cor como YCbCr.

\subsubsection{Extração de características}

Antes de realizar o treinamento efetivo do modelo, será necessário extrair características a partir das imagens médicas, como textura, bordas e padrões. Durante a fase de extração de características, o principal objetivo é transformar os dados recebidos em representações que melhor capturam os padrões subjacentes e são mais informativas para o algoritmo de aprendizado. Para a seleção de características relevantes, técnicas como análise de correlação e análise de variância serão utilizados para apoio à seleção. Para os dados de imagens, a extração de características estará incorporada ao treinamento, visto que as camadas convolucionais têm a responsabilidade de aprender os padrões hierárquicos e abstratos.

\subsubsection{Treinamento do modelo}

Com todas as informações já preparadas, serão exploradas diferentes técnicas de convoluções para a segmentação das imagens microscópicas, como por exemplo a \textit{pixel-wise convolution}, uma técnica que possibilita a combinação de informações entre os canais de uma imagem. Será avaliado a possibilidade de utilizar técnicas de \textit{ensemble} nessa fase, de modo a combinar da melhor maneira os dados clínicos com as imagens microscópicas. Também será explorado técnicas para otimizar o treinamento, como ajuste da taxa de aprendizado de forma dinâmica entre as diferentes camadas de acordo com algumas métricas, como o erro das previsões. Para fins de melhora no desempenho, o treinamento do modelo também envolverá técnicas de computação paralela, com isso será possível um treinamento mais otimizado do sistema.

\subsubsection{Avaliação do modelo}

Como etapa final, a avaliação do modelo é responsável por aplicar métodos estatísticos para consolidar os resultados das etapas previstas. Conforme os resultados desses métodos, será possível analisar se o desempenho do modelo está de acordo com o esperado. Dentre as técnicas mais comuns, serão utilizadas métricas como acurácia, definida pela fórmula 
\begin{equation}
  acurácia = \frac{VP +VN}{VP + FN + VN + FP} \; ,
    \label{eq: acuracia}
\end{equation}
\\
em que \(VP\) é a quantidade de verdadeiros positivos; \(VN\) é a quantidade de verdadeiros negativos; \(FN\) é a quantidade de falsos negativos e \(FP\) é a quantidade de falsos positivos. Também será utilizado a métrica de avaliação precisão, determinada pela fórmula:

\begin{equation} 
  precisão = \frac{VP}{VP + VF} \; ,
  \label{eq: precisao}
\end{equation}
 \\
E por último, uma métrica de avaliação chamada sensibilidade, estabelecida pela fórmula:

\begin{equation}
sensibilidade = \frac{VP}{VP + FN} \; ,
  \label{eq: sensibilidade}
\end{equation}
\\
Paralelamente, todas as fases irão contar com um minucioso registro de todas as ações para monitoramento do sistema. Além disso, serão implementados mecanismos de auditoria para garantir a rastreabilidade das operações realizadas, promovendo maior transparência e confiabilidade. Esse registro incluirá não apenas informações sobre as decisões tomadas pelo sistema, mas também os dados utilizados em cada etapa, possibilitando a análise detalhada de possíveis erros ou inconsistências e facilitando ajustes futuros para otimização do desempenho.

\subsection{Do protótipo da aplicação web}

Para o uso desse modelo em um potencial sistema, serão utilizados diferentes componentes de tecnologia, envolvendo áreas como arquitetura de software, desenvolvimento backend, computação em nuvem, e conceitos de computação distribuída, a fim de projetar um sistema que seja resiliente, distribuído e escalável.

\subsubsection{Da arquitetura do protótipo}
O sistema proposto adota uma arquitetura baseada em microsserviços, modelo arquitetural que promove o desacoplamento dos componentes, permitindo evolução independente de cada serviço. Essa abordagem facilita a escalabilidade horizontal, melhora a manutenção e possibilita a implementação de diferentes tecnologias conforme as necessidades específicas de cada módulo. Além disso, a divisão funcional dos serviços contribui para um gerenciamento mais eficiente da infraestrutura e do processamento de dados, garantindo maior flexibilidade e modularidade no desenvolvimento e na implantação.

\subsubsection{Da definição dos protocolos de comunicação}
Além da definição da arquitetura desse protótipo, também serão explorados protocolos de comunicação de alta desempenho para garantir maior confiabilidade e reduzir latência entre os serviços. Protocolos como o \textit{Remote Procedure Call} (RPC) serão utilizados para garantir essa comunicação eficiente entre serviços. O RPC possui algumas vantagens em relação a outros protocolos, dentre elas estão a serialização binária e suporte a conexões persistentes, características que garantem um \textit{payload} menor sendo trafegado, reduzindo a latência e otimizando o desempenho \cite{Niswar2024}. 

\subsubsection{Do monitoramento do protótipo}
A fim de avaliar o desempenho do sistema proposto, serão implementadas estratégias de monitoramento para análise detalhada do comportamento da aplicação em diferentes cenários de carga e uso. Ferramentas como Prometheus e Grafana serão empregadas para coleta de métricas e visualização do desempenho dos microsserviços em tempo real, permitindo identificação de potenciais gargalos e otimizações. Além disso, serão adotadas técnicas de observabilidade avançadas, como logs estruturados e rastreamento distribuído com OpenTelemetry, possibilitando uma análise aprofundada das interações entre os serviços e assegurando uma resposta eficiente a eventuais falhas.

% --- Poderia ficar nos objetivos? ---
% Além do desenvolvimento de um modelo que seja capaz de segmentar imagens microscópicas de células com alta precisão, esse trabalho também propõe a aplicação desse modelo num sistema destinado a uso médico. 

