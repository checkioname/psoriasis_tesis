\section{Conclusão}

Este trabalho apresentou o desenvolvimento de um sistema \gls{cad} para diagnóstico de psoríase vulgar baseado em \textit{Visual Transformers}, utilizando dados de pacientes brasileiros fornecidos pela \gls{fmabc}. O objetivo principal foi criar uma ferramenta de apoio ao diagnóstico capaz de distinguir entre psoríase vulgar e dermatite por meio de técnicas de visão computacional e processamento digital de imagens. Para atingir essa meta, foi implementado um \textit{pipeline} completo que envolveu desde a coleta e curadoria dos dados até o treinamento e avaliação de um modelo baseado na arquitetura \textit{Swin Transformers}. O modelo desenvolvido alcançou resultados promissores, com acurácia de 93,78\%, precisão de 92,11\%, revocação de 91,42\% e F1-Score de 91,76\%, demonstrando sua capacidade de classificação binária entre as duas condições dermatológicas estudadas.

Apesar dos resultados quantitativos satisfatórios, é fundamental destacar as limitações inerentes ao modelo, em grande parte, devido às características do dataset utilizado. A redução de mais de 50\% da base de dados original, embora necessária para garantir a qualidade das amostras, resultou em um volume limitado de imagens para o treinamento. 

Adicionalmente, as imagens foram capturadas em um ambiente controlado de laboratório universitário, o que as torna menos representativas de cenários clínicos reais, onde a variação de iluminação, qualidade da câmera e angulação é muito maior. Consequentemente, o modelo pode ter sua capacidade de generalização comprometida ao ser aplicado a imagens com características diferentes daquelas do conjunto de dados curado, como em um cenário de triagem em consultórios médicos. Tais limitações reforçam a necessidade de futuras pesquisas que utilizem um conjunto de dados mais diversificado, de maior volume e que simulem as condições do mundo real, a fim de aprimorar a robustez e a aplicabilidade clínica do modelo.

Apesar das limitações inerentes ao conjunto de dados, os resultados obtidos nesta pesquisa demonstram o potencial significativo do desenvolvimento de um sistema \gls{cad}.
A utilização de dados de pacientes brasileiros confere um valor estratégico ao modelo para a aplicabilidade em cenários clínicos nacionais. Além disso, este estudo serve como um ponto de partida para trabalhos futuros promissores. A evolução para um sistema capaz de realizar a segmentação automática da lesão ou a classificação da severidade da doença são direções de pesquisa que poderiam ampliar a funcionalidade e o impacto clínico da ferramenta.

