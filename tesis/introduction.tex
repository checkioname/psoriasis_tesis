\section{Introdução}
A relação entre a computação e área da saúde está cada vez mais próxima, prova disso são as diversas aplicações da computação na medicina, desde plataformas de atendimento remoto (telemedicina) até sistemas de diagnóstico guiado por computador (\gls{cad}).
Os sistemas \gls{cad} podem ser compostos de diferentes técnicas da computação. Em pesquisas recentes, esses sistemas têm sido beneficiados pelo uso da \ac{IA}, uma área da computação que tem a intenção de replicar a inteligência humana. Uma subárea da \acs{IA} importante para os sistemas \gls{cad} é a Visão Computacional, cuja meta é utilizar computadores para emular a visão humana, incluindo o aprendizado e a capacidade de fazer inferências e agir com base em informações visuais. Uma área que age em conjunto com a Visão Computacional é o Processamento Digital de Imagens, que geralmente está atribuída no estudo e aplicação de técnicas de mais baixo nível, como o realce de contraste e aguçamento de imagens \cite{gonzalez2008digital}.

Na dermatologia, existe uma lacuna entre pacientes de doenças de pele e a expertise necessária para lidar com eles \cite{Hameed2019}. Pessoas que vivem nas áreas rurais são as que mais se prejudicam por conta da falta de recursos, aponta pesquisa da Organização Mundial de Saúde (2015). Os sistemas \gls{cad} são muito vantajosos nesses cenários, oferecendo um pré-diagnóstico de diversas doenças. Além disso, esses sistemas buscam proporcionar ao processo de definição da doença maior precisão, garantindo o tratamento adequado ao paciente e diminuindo custos operacionais (\cite{Hameed2019}, \cite{DASH2020106240}, \cite{Arora2021}. Uma doença dermatológica que pode ser analisada e diagnosticada por sistemas \gls{cad} é a psoríase. A psoríase é uma doença inflamatória, crônica e recorrente que, de acordo com a Sociedade Brasileira de Dermatologia (2021), afeta a pele e articulações de mais de 5 milhões de brasileiros.

Existem vários fenótipos dessa doença, sendo a mais comum a psoríase vulgar, presente em cerca de 90\% dos casos \cite{Griffiths2007}. Essa doença não apresenta risco à vida diretamente, porém traz diversas outras implicações para o paciente, desde coceira até sangramento na região lesionada. Estudos epidemiológicos identificaram também uma alta prevalência de fatores de risco cardiovascular em pacientes psoriáticos. Essa característica é particularmente importante porque, as doenças cardiovasculares -- incluindo a síndrome metabólica, obesidade, hipertensão, \textit{diabetes mellitus}, resistência à insulina e a dislipidemia \cite{Miller2013} -- são a principal causa de morte, hospitalizações e atendimentos ambulatoriais em todo o mundo, inclusive em países em desenvolvimento como o Brasil \cite{Barroso2021}.

% Existem vários fenótipos dessa doença, sendo a mais comum a psoríase vulgar, presente em cerca de 90\% dos casos \cite{Griffiths2007}. Essa doença não apresenta risco à vida diretamente, porém traz diversas outras implicações para o paciente, desde coceira até sangramento na região lesionada. Estudos epidemiológicos identificaram também uma alta prevalência de fatores de risco cardiovascular em pacientes psoriáticos, característica importante, já que as doenças cardiovasculares como a síndrome metabólica, obesidade, hipertensão, \textit{diabetes mellitus}, resistência à insulina e a dislipidemia \cite{Miller2013} são a principal causa de morte, hospitalizações e atendimentos ambulatoriais em todo o mundo, inclusive em países em desenvolvimento como o Brasil \cite{Barroso2021}. 

A psoríase tem seu diagnóstico realizado inicialmente através de uma análise visual, ou seja, pela aparência clínica e distribuição da lesão. Identificada, ela é classificada como leve, moderada ou grave, sendo essa classificação baseada principalmente na superfície corporal afetada e no efeito da lesão no paciente. Diante disso, uma medida para a classificação da doença é a \gls{pasi}, que é dividida em dois passos, o primeiro é o cálculo da Área de superfície corporal (BSA) afetada com as lesões e o segundo passo consiste em avaliar o eritema (vermelhidão na pele), o endurecimento (espessura) e a descamação das lesões (Sociedade Brasileira de Dermatologia, 2018).

\subsection{Objetivos}
O presente projeto tem como objetivo conduzir um estudo motivado pelo alto número de pacientes afetados pela psoríase, e também oportunidades de utilização de dados de pacientes brasileiros, para desenvolver um sistema \gls{cad} dermatológico, baseado em um modelo de \acs{IA} utilizando técnicas de visão computacional, processamento digital de imagens para diagnóstico de psoríase vulgar. Este trabalho dará início ao desenvolvimento de um sistema \gls{cad} próprio,  utilizando imagens e dados clínicos de pacientes brasileiros. Essa característica é um dos pontos chaves do estudo, já que devido a fatores como a região, clima e exposição solar, a amostragem local é muito diferente comparada a outros trabalhos já propostos. Para atingir essa meta foram estipulados os seguintes objetivos específicos:
\begin{itemize}
  \item Definir e coletar as bases para o treinamento do modelo;
  \item Definir a arquitetura e treinar um modelo de \acs{IA} para a classificação entre psoríase vulgar e dermatite; usando imagens dermatológicas;
  \item Validar os resultados de desempenho do modelo treinado;
  \item Analisar a viabilidade técnica de aplicação do método proposto em um produto real.

\end{itemize}
