\section{Introdução}

Nos últimos anos, as aplicações da tecnologia na área da saúde tiveram um grande crescimento, marcadas por inovações relevantes que contribuíram para o tratamento clínico de diversas doenças. Hoje em dia vemos essa relação cada vez mais próxima, representada pelo desenvolvimento desde plataformas de atendimento remoto (telemedicina), dispositivos vestíveis como sensores biométricos até sistemas de diagnóstico guiado por computador (computer aided diagnosis (CAD)). Os sistemas CAD são sistemas que tem a intenção de auxiliar no diagnóstico clínico, estes sistemas podem ser compostos de diferentes técnicas da computação. Pesquisas recentes, como o trabalho de \citeonline{BRINKER201911}, apontam que uma das técnicas que tem mostrado resultados relevantes para os sistemas CAD é o uso de Inteligência Artificial (IA), um campo da computação que tem a intenção de replicar a inteligência humana. 

Mesmo com os avanços tecnológicos na saúde, o câncer ainda é uma das causas mais comuns de morte em todo o mundo, causando por volta de 10 milhões de mortes em 2021 (Global Burden of Disease, 2024). Essa doença é causada por células desviantes, incapazes de formar estruturas funcionais estáveis, que se multiplicam anarquicamente e invadem o organismo \cite{FLOOR2012509} . Sendo que, a maioria dos casos de morte em tratamentos de câncer está relacionado com a metástase, o processo pelo qual a célula cancerígena escapa do tumor primário e entra na corrente sanguínea, a fim de se disseminarem para outros lugares secundários por no corpo \cite{THAM201181}. De acordo com o Instituto Nacional de Câncer (INCA) (2023), a estimativa de ocorrência de novos casos de câncer até 2030, no Brasil são de mais de 25 milhões de casos. Diante disso, os sistemas de diagnóstico automatizado desempenham um papel importante para o aumento da eficiência no diagnóstico da doença, além de promover maior agilidade operacional, principalmente em ambientes onde há escassez de profissionais da área da saúde e a espera por atendimento é longa.

A ocorrência de metástase no sistema linfático é muito comum em alguns tipos de câncer, como o câncer colorretal e câncer de estomago \cite{Sano1992}. Atualmente, as estratégias para lidar com a metástase envolvem radioterapia e ectomia ou amputação cirúrgica.
Para compensar as taxas de mortalidade associadas à doença metastática, é fundamental combater os pequenos tumores antes que estes se transformem em malignidade irrestrita e/ou progridam ainda mais \cite{DAS2020119556}. Para isso, o diagnóstico e tratamento precoce são os fatores principais e se tornam ainda mais essenciais.
Shaw (2000) diz que para um paciente com câncer, deve ser dado peso igual ao tratamento das lesões tumorais primárias e da metástase, mas a supressão da metástase é mais árdua.

No contexto de diagnóstico clínico, a análise de exames de imagem da área lesionada é muito comum, sob o mesmo ponto de vista, há uma subárea da IA que tem como objetivo explorar algoritmos de modo a emular a visão humana nos computadores, incluindo o aprendizado e a capacidade de fazer inferências e agir com base em informações visuais, essa subárea é denominada Visão Computacional. Outro domínio que age em conjunto com a Visão Computacional é o Processamento Digital de Imagens, que geralmente está atribuído no estudo e aplicação de técnicas de mais baixo nível, como o realce de contraste e aguçamento de imagens \cite{gonzalez2008digital}.

O presente projeto tem como objetivo conduzir um estudo motivado pelo alto número de pacientes afetados pelo câncer, pelas dificuldades de acompanhamento da metástase, e também oportunidades de utilização de dados de dispositivos microscópicos, para desenvolver um sistema CAD, baseado em um modelo de IA utilizando técnicas de visão computacional, processamento digital de imagens e dados clínicos para diagnóstico de câncer. Este trabalho dará início ao desenvolvimento de um sistema CAD, responsável pela análise unicelular do comportamento de células cancerígenas afim de prever comportamento metastático utilizando imagens e dados clínicos públicos de dispositivos microscópicos. Essa característica é um dos pontos chaves do estudo, já que atualmente essa combinação de dados não está presente em pesquisas anteriores, almejando assim atingir resultados superiores e mais consistentes para esse público. Para atingir essa meta serão trabalhados os seguintes objetivos específicos:

\begin{itemize}
        \item Definir e coletar as bases para o treinamento do modelo;
        \item Definir a arquitetura e treinar um modelo de IA para a classificação de metástase em células cancerígenas usando imagens microscópicas;
        \item Validar os resultados de desempenho do modelo treinado;
        \item Definir a arquitetura e treinar um modelo de IA para a classificação usando os dados clínicos, que também devem compor o resultado da classificação das imagens;
        \item Validar os resultados de desempenho do modelo de dados clínicos;
        \item Analisar o ganho de desempenho com a combinação de imagens e dados clínicos;
        \item Construir uma pipeline integrando os dois modelos;
        \item Analisar a viabilidade técnica de aplicação do método proposto em um produto real.
\end{itemize}

A implementação do sistema CAD proposto promoverá maior democratização ao acesso a diagnósticos de alta precisão, principalmente em regiões com escassez de profissionais. Com isso, também contribuirá para maior eficiência operacional e irá favorecer o diagnóstico precoce da doença, sendo ponto inicial para a melhora de chances de sobrevida e melhora na qualidade de vida do paciente. Para a indústria da saúde, o sistema tem potencial de aumentar a precisão dos diagnósticos e chances de detectar estágios recentes da doença, por meio de características não observáveis a olho nu, logo, minimizando possíveis riscos de falhas humanas. A fim de validar os resultas obtidos, esse trabalho também propõe implementar um protótipo de um sistema web que embarca o modelo e permite a interação de um usuário com esse agente.

Para a apresentação do trabalho, este documento foi estruturado da seguinte forma.
A seção 2, o referencial teórico, contém os levantamentos, especificações e revisões de outros trabalhos propostos na área, com ênfase nos modelos de diagnóstico por imagem e algoritmos de aprendizado de máquina. Na seção 3, são descritos os materiais e métodos utilizados, incluindo a coleta de dados, as técnicas de pré-processamento das imagens e o desenvolvimento do modelo proposto. A seção 4 estabelece o cronograma do trabalho, detalhando as etapas realizadas e as previstas para futuras implementações e validações.
