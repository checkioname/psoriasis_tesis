\section{Resultados e Discussão}

A rede treinada apresentou resultados relevantes e muito promissores. Um dos principais desafios encontrado durante o processo de desenvolvimento foi a complexidade de trabalhar com o dataset, além do pré-processamento, també foi necessário a remoção de certas imagens que não estavam adequadas para o treinamento da rede, devido a diversos fatores externos, como por exemplo qualidade da foto. Mesmo em um ambiente controlado não havia um padrão da maneira como os dados eram gerados, portanto diversas imagens necessitavam de análise individual. A tabela \ref{tab:db-table} indica algumas informações sobre o dataset utilizado para o treinamento do modelo após o pré-processamento. 

Após todo o tratamento do dataset houve uma diminuição expressiva do volume da base de dados, um pouco mais de 50\% de redução, esse fator também gerou outros desafios durante a fase de treinamento, como a falta de elementos representativos. A imagem abaixo é um exemplo do resultado final sendo alimentado na rede.

\newline
\begin{table}[h]
  \centering
  \begin{tabular}{lll}
  \hline
  \multicolumn{1}{|l|}{No} & \multicolumn{1}{l|}{Label} & \multicolumn{1}{l|}{Quantidade} \\ \hline
  1                        & Psoriasis                  & 1019                            \\
  2                        & Dermatite                  & 1239                            \\
  \end{tabular}
  \caption{Tabela }
  \label{tab:db-table}
\end{table}


Durante o treinamento e a avaliação foram utilizadas diferentes técnicas que possibilitassem a análise do desempenho do modelo, a figura demostra as áreas de interesse que o modelo analisou para a classificação da doença. A partir dessa figura, é possível compreender como o modelo foi capaz de relacionar o contexto da região vermelha (área lesionada) com a classificação da doença psoriase


\newpage


