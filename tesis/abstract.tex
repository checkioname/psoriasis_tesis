% RESUMO
\thispagestyle{plain}

\vspace{1.0cm}
   
\renewcommand{\baselinestretch}{0.6666666}
{\bf Resumo.} Câncer é a principal causa de mortes em todo o mundo. A dificuldade de identificar a doença é um dos principais fatores que leva a esse desfecho. Entretanto com os avanço dos algoritmos de visão computacional, a classificação automatizada de doenças atrvés desses algoritmos tem se mostrado muito positiva. Assim, o presente projeto tem como objetivo o desenvolvimento de um sistema de diagnóstico assistido por computador para o auxílio na detecção precoce de câncer, com enfoque em aplicações baseadas em inteligência artificial e análise de imagens médicas. O objetivo principal é propor e implementar um pipeline de aprendizado de máquina que integre técnicas de pré-processamento, engenharia de características e modelos de aprendizado profundo para a identificação e classificação de anomalias em imagens microscópicas, visando aumentar a precisão diagnóstica e reduzir a variabilidade entre especialistas.
\begin{flushleft}
{\bf Palavras-chave:} {\it CAD, Neural Networks, CAD Microscopy, CNN, CAD intravital}
\\[2.5cm]
\end{flushleft}


{\bf Abstract.} Cancer is the leading cause of deaths worldwide. The difficulty in identifying the disease is one of the main factors that leads to this outcome. However, with advances in computer vision algorithms, the automated classification of diseases through these algorithms has proven to be very positive. Thus, the present project aims to develop a computer-assisted diagnosis system to aid in the early detection of cancer, focusing on applications based on artificial intelligence and medical image analysis. The main objective is to propose and implement a machine learning pipeline that integrates pre-processing techniques, feature engineering and deep learning models for the identification and classification of anomalies in microscopic images, aiming to increase diagnostic accuracy and reduce variability between experts.
\begin{flushleft}
{\bf Keywords:} {\it CAD, Neural Networks, CAD Microscopy, CNN, CAD intravital}
\end{flushleft}
